\documentclass[11pt]{article}
\usepackage{a4wide}

\begin{document}

\section{Tage datafile description}

This file describes tage input files which describes whole scene.

\section{Comments}

The input file uses standard C++ comments:

\begin{verbatim}
// One line comment

/*
  Two or more line comment
*/
\end{verbatim}

\section{Properties}

All properties are set by:

\begin{verbatim}
 property_name = value
\end{verbatim}

a "value" can be strings, numbers (hexa, integer, float-point), colors, vectors
and enumerated values.

\subsection{Numbers}

Numbers are standard numerical values and can have a decimal part.

\begin{verbatim}
 size = 10
 height = 1.1
\end{verbatim}

\subsection{Strings}

Strings don't use commas and can't contain spaces. String values are typically 
used for modificator/generator names.

\begin{verbatim}
 name = my_modificator_name
\end{verbatim}

\subsection{Colors}

Colors can be defined by who ways - by separated RGB values,
(0-255) by one hexadecimal digit (HTML color) or as a vector (R,G,B). 
For instance, we want to set R:33, G:25, B:7 to color\_center value:

\begin{verbatim}
 // by RGB:
 color_center_r = 33
 color_center_g = 25
 color_center_b = 7
 
 // by one hexa number (RRGGBB)
 color_center = 211907
 
 // by vector (R,G,B)
 color_center = (33,25,7) 
\end{verbatim}

See the \_r,\_g and \_b suffixes. They are 0 by default.

\subsection{Vectors}

Vectors are composed from two or three numbers and they
can be integer or floating point numbers. For instance, we want to 
set light\_position vector:

\begin{verbatim}
 light_position_x = -1
 light_position_y =  1
 light_position_z = -1
\end{verbatim}

See the \_x,\_y and \_z suffixes. They are 0 by default. Another option is
to use a vector format (x,y,z):

\begin{verbatim}
 light_position = (-1,1,-1)
\end{verbatim}

\subsection{Angles}

An angles are normal numbers (an angle in degrees), from 0 to 360. 
They are used in polar coordinates and so on.

\begin{verbatim}
 some_angle = 20.6
\end{verbatim}

\subsection{Enumerated types}

Enumerated types are values which can have some predefined values. 
They are typically used for blocks type descriptions, some types,
targets and so on.

\begin{verbatim}
// coordinate type
type = MODIFICATOR_COORDINATE

// set modificator_target to texture
modificator_target = TEXTURE

// set modificator_target to geometry
modificator_target = GEOMETRY
\end{verbatim}

\subsection{Intervals}

Some values can be set as interval. If a value is an interval,
it means it can get any value from the border values. The border values
are marked as "\_min" and "\_max" suffixes. Intervals are always used
with other types (number, angle, color, vector). Intervals can be set 
as a normal (non-interval) value, too.

\begin{verbatim}
 /* Number intervals
 */
 // Interval set by only one value so it's always 10
 angle = 10

 // Interval set by two border values, 
 // can be any value from 10 to 20
 angle_min = 10
 angle_max = 20 
 
 /* Vector intervals
 */ 
 // As components
 position_min_x = 10
 position_min_y = 10
 position_min_z = 10
 
 position_max_x = 20
 position_max_y = 20
 position_max_z = 20

 // As vectors
 position_min = (10,10,10)
 position_max = (20,20,20)

 /* Color intervals
 */
 // As components
 color_min_r = 10
 color_min_g = 10
 color_min_b = 10
 
 color_max_r = 20
 color_max_g = 20
 color_max_b = 20
 
 // As vectors
 color_min = (10,10,10)
 color_max = (20,20,20)
 
 // As hexadecimal (HTML) colors
 color_min = 0a0a0a
 color_max = 141414

\end{verbatim}

\subsection{Coordinates}

Coordinates are 2D area which describes 
where a modificator is applied.  The coordinate is a whole 
block with "type = MODIFICATOR\_COORDINATE", index (will be described later)
and start and size (or end) 2D vectors.

\begin{verbatim}
{
  type = MODIFICATOR_COORDINATE

  index = 0

  start_x = 0
  start_y = 0

  size_x = 40
  size_y = 40
}
\end{verbatim}

This example describes an area which begins at (0,0) and size 40x40 pixels.

\section{Basic blocks}

An atomic part of the file is a block inside compound braces. It describes one 
atomic unit inside generator or some generator values. Each block must
contain its name and type.

\begin{verbatim}
{
  name = generator
  type = GENERATOR_MESH

  /* All generator params come here
  */
}
\end{verbatim}

Blocks can be nested, like this one:

\begin{verbatim}
/* Describes pixel generator and its color definition
*/
{
  name = pixel_point
  type = MODIFICATOR_POINT_SINGLE

  {
    type = MODIFICATOR_POINT_SINGLE_COLOR
    color_center = 3b5528
  }
}
\end{verbatim}

All block examples bellow uses this format:

\begin{verbatim}
{
  /* First part contains block name and type:
  */
  name = block_name
  type = block_type

  /*
    Second part is a list of all posible properties,
    descriptions and default values:

    [property_type]  property_name  - property description
   
    If the property_type is an enumerated type, all 
    posibilies come here:
    
    VALUE_1     - a description of VALUE_1
    VALUE_2     - a description of VALUE_1
    VALUE_3     - a description of VALUE_1
  */  
  property_name = default_value_of_the_property  
}
\end{verbatim}

\section{Generator architecture}

Whole generator is designed as a modificator chain. There is one master (root)
modificator and it passes results to slave modificators. A last modificator 
in the chain writes results (color pixel, heights) directly to a generator target
(it can be mesh itself, mesh texture or something else).

\paragraph{Basic terms:}
\subparagraph{Modificators}
are atomic generator parts specialized to one task. Each modificator 
is configured by properties (from the data file), coordinates (from the data 
file and/or previous modificator) and parameters (from previous modificator) and
pastes its results (coordinates, properties, parameters) to another modificator. 

For instance there is a modificator which generates a line and it 
pastes the results (coordinates for each single point which lies on the line) 
to another modificator which draws them.

\subparagraph{Targets}
are "final" modificators which transforms the results to geometry (mesh) or texture. 

\subparagraph{Generator}
launches one or more modificators and specify which targets are used.  
An output of generator is a complete 3D object with material and texture. 

Generator itself can be used as a modificator so
if we take the line modificator from previous example, 
the line modificator -> pixel modificator -> texture target chain will 
generate single pixels to texture, but line modificator -> generator 
chain will generate complete 3D objects on given coordinates.

\subparagraph{Generator launcher}
launches generators.

\section{Generators}

\subsection{Generator launcher}

Generator launcher defines which generators are performed and their order. 
It can be only one in the whole data file.

\begin{verbatim}
{
  /* Launcher name and type
  */
  name = generator_launcher_name
  type = GENERATOR_LAUNCHER

  /* Performed generators. 
  */
  generator_mesh = first_generator
  generator_mesh = second_generator
  generator_mesh = third_generator
}
\end{verbatim}

\subsection{Generator}

Generator defines which modifiators are launched, their targets and order. 
There can be as many generators as you want in a data file and 
are distinguished by their names:

\begin{verbatim}
/* A simple generator
*/
{
  /* Generator name and type
  */
  name = generator_name
  type = GENERATOR_MESH

  /*
    Modificator name and its target:
    
    [string]          modificator - a name of performed modificator
    [enumerated type] modificator_target - its target
    
    Modificator targets can be:
    
    TEXTURE         - texture target (color or height)
    GEOMETRY        - heights in mesh geometry
    GENERATOR_MESH  - target is another generator
    AUX             - an auxiliary surface (color or height)
  */
}
\end{verbatim}

There is an example of some generator there: 

\begin{verbatim}
/* A simple generator
*/
{
  /* Generator name and type
  */
  name = generator_name
  type = GENERATOR_MESH

  /* First modificator name and its target
  */
  modificator = first_modificator
  modificator_target = TEXTURE
  
  /* Second modificator name and its target
  */
  modificator = second_modificator
  modificator_target = GEOMETRY
}
\end{verbatim}

\subsection{Generated object parameters}

A 3D object generated by single generator is (for now) a flat mesh with
one big texture. If the texture is too big, it's sliced to smaller parts.
The object is described by mesh, material and texture block.

\subsubsection{Mesh params}

Describes generated mesh parameters like type, size and so on:

\begin{verbatim}
{
  name = mesh_name
  type = MESH_PARAMS

  /*
    [enumerated value]  mesh_type
  
    Mesh types can be:
   
    MESH_LAND   - a flat land
    MESH_BUNCH  - a bunch of plates
    MESH_GRASS  - not implemented yet
    MESH_BUSH   - not implemented yet
  */
  mesh_type = MESH_LAND
  
  /*
    Mesh dimensions. All values are vectors.
  
    [vector] start - lesh location 
    [vector] diff  - a size of one segment 
    [vector] size  - segments num
  */  
  start = (0,0,0)
  diff = (1,1,1)
  size = (1,1,1)
  
  /*
    Parameters related to bunch:
    
    [int, interval]   bunch_slice_num
    [int, interval]   bunch_slice_segments
      
    [float, interval] bunch_slice_x_offset
    [float, interval] bunch_slice_z_offset
      
    [angle, interval] bunch_slice_falling
    [angle, interval] bunch_segment_falling
        
    [int]             bunch_slice_rotation_incemental
    [angle, interval] bunch_slice_rotation_range
    [angle, interval] bunch_slice_rotation_step
  */    
  bunch_slice_num = 6
  bunch_slice_segments = 1
  
  bunch_slice_x_offset = 0
  bunch_slice_z_offset = 0
  
  bunch_slice_falling = 0
  bunch_segment_falling = 0
  
  bunch_slice_rotation_incemental = 0
  bunch_slice_rotation_range = 180
  bunch_slice_rotation_step = 0
}
\end{verbatim}

\subsubsection{Material params}

Describes material of a generated mesh:

\begin{verbatim}
{
  name = test_material
  type = MATERIAL_PARAMS
  
  // [int] transparent
  // Transparent material are for bunches
  transparent = 0
  
  // [int] double_side
  // Double sided material are used by bunches
  double_side = 0  
}
\end{verbatim}

\subsubsection{Texture params}

Describes texture for a generated mesh.

\begin{verbatim}
{
  name = test_texture
  type = TEXTURE_PARAMS

  /*
    [vector]  texture_size
    [int]     texture_height
    [color]   background_color
    [int]     texture_alpha
  */
  texture_size = (512,512)  
  texture_height = 512
  background_color = (0,0,0)
  texture_alpha = 0
}
\end{verbatim}

\section{Generator targets}
\subsection{GEOMETRY target}
\subsection{TEXTURE target}
\subsection{GENERATOR\_MESH target}
\subsection{AUX target}

\section{Generator modificators}

\subsection{A generic modificator}
\begin{verbatim}
{

}
\end{verbatim}

\subsection{Coordinate specification}
MODIFICATOR\_COORDINATE

\subsection{Point modificators}
\subsubsection{Single point modificator}
MODIFICATOR\_POINT\_SINGLE
MODIFICATOR\_POINT\_SINGLE\_COLOR
MODIFICATOR\_POINT\_SINGLE\_HEIGHT
\subsubsection{Extended point modificator}
MODIFICATOR\_POINT\_EXTENDED

\subsection{Rectangle modificator}
MODIFICATOR\_RECT

\subsection{Height modificators}
\subsubsection{Height map modificator}
MODIFICATOR\_HEIGHT\_MAP
\subsubsection{Mid-point modificator}
MODIFICATOR\_FRACTAL
\subsubsection{Perlin noise modificator}
MODIFICATOR\_PERLIN

\subsection{Line modificators}
\subsubsection{Single line modificator}
MODIFICATOR\_LINE
\subsubsection{Leaf modificator}
MODIFICATOR\_LINE\_LEAF
\subsubsection{Crack modificator}
MODIFICATOR\_CRACK
\subsubsection{Network modificator}
MODIFICATOR\_NET

\subsection{Bunch modificator}
MODIFICATOR\_BUNCH

\subsection{Mask modificator}
MODIFICATOR\_MASK

MODIFICATOR\_BITMAP
MODIFICATOR\_LIGHT
MODIFICATOR\_FILTER
MODIFICATOR\_GENERATOR\_MESH

\end{document}
