\documentclass[9pt]{article}
\usepackage{a4wide}

\begin{document}

\section{Tage datafile description}

This file describes tage input files which describes whole scene.

\section{Comments}

The input file uses standard C++ comments:

\begin{verbatim}
// One line comment

/*
  Two or more line comment
*/
\end{verbatim}

\section{Properties}

All properties are set by:

\begin{verbatim}
 property_name = value
\end{verbatim}

a "value" can be strings, numbers (hexa, integer, float-point), colors, vectors
and enumerated values.

\subsection{Boolean}

Boolean is a binary value (true/false) and it's used for switches
or on/off properties. The false value is written as 0 and true any other
number (typically 1).

\begin{verbatim}
 // light is enabled 
 enable_ligth = 1
 
 // shadows are disabled
 enable_shadows = 0
\end{verbatim}

\subsection{Numbers}

Numbers are standard numerical values and can have a decimal part.

\begin{verbatim}
 size = 10
 height = 1.1
\end{verbatim}

\subsection{Strings}

Strings don't use commas and can't contain spaces. String values are typically 
used for modificator/generator names.

\begin{verbatim}
 name = my_modificator_name
\end{verbatim}

\subsection{Colors}

Colors can be defined by who ways - by separated RGB values,
(0-255) by one hexadecimal digit (HTML color) or as a vector (R,G,B). 
For instance, we want to set R:33, G:25, B:7 to color\_center value:

\begin{verbatim}
 // by RGB:
 color_center_r = 33
 color_center_g = 25
 color_center_b = 7
 
 // by one hexa number (RRGGBB)
 color_center = 211907
 
 // by vector (R,G,B)
 color_center = (33,25,7) 
\end{verbatim}

See the \_r,\_g and \_b suffixes. They are 0 by default.

\subsection{Vectors}

Vectors are composed from two or three numbers and they
can be integer or floating point numbers. For instance, we want to 
set light\_position vector:

\begin{verbatim}
 light_position_x = -1
 light_position_y =  1
 light_position_z = -1
\end{verbatim}

See the \_x,\_y and \_z suffixes. They are 0 by default. Another option is
to use a vector format (x,y,z):

\begin{verbatim}
 light_position = (-1,1,-1)
\end{verbatim}

\subsection{Angles}

An angles are normal numbers (an angle in degrees), from 0 to 360. 
They are used in polar coordinates and so on.

\begin{verbatim}
 some_angle = 20.6
\end{verbatim}

\subsection{Enumerated types}

Enumerated types are values which can have some predefined values. 
They are typically used for blocks type descriptions, some types,
targets, operations and so on.

\begin{verbatim}
// coordinate type
type = MODIFICATOR_COORDINATE

// set modificator_target to texture
modificator_target = TEXTURE

// set modificator_target to geometry
modificator_target = GEOMETRY
\end{verbatim}

\subsubsection{Aritmetic operation}

It's one of frequently applied enumerated types and defines requested arithmetics
operation. It's used for coordinates, color/height operations and many more.

\begin{verbatim}  
  
  Aritmetic operation anumerator is used in this context:
  
  result = destination OP source
  
  and OP is defined as:

  SET           result = source
  ADD           result = destination + source
  SUB           result = destination - source
  MODULATE      result = destination * source
  MODULATE2X    result = destination * source * 2
    
\end{verbatim}

\subsection{Intervals}

Some values can be set as interval. If a value is an interval,
it means it can get any value from the border values. The border values
are marked as "\_min" and "\_max" suffixes. Intervals are always used
with other types (number, angle, color, vector). Intervals can be set 
as a normal (non-interval) value, too.

\begin{verbatim}
  /* Number intervals
  */
  // Interval set by only one value so it's always 10
  angle = 10
  
  // Interval set by two border values, 
  // can be any value from 10 to 20
  angle_min = 10
  angle_max = 20 
  
  /* Vector intervals
  */ 
  // As components
  position_min_x = 10
  position_min_y = 10
  position_min_z = 10
  
  position_max_x = 20
  position_max_y = 20
  position_max_z = 20
  
  // As vectors
  position_min = (10,10,10)
  position_max = (20,20,20)
  
  /* Color intervals
  */
  // As components
  color_min_r = 10
  color_min_g = 10
  color_min_b = 10
  
  color_max_r = 20
  color_max_g = 20
  color_max_b = 20
  
  // As vectors
  color_min = (10,10,10)
  color_max = (20,20,20)
  
  // As hexadecimal (HTML) colors
  color_min = 0a0a0a
  color_max = 141414
  
\end{verbatim}

\subsection{Coordinates}

Coordinates are 2D area which describes 
where a modificator is applied.  The coordinate is a whole 
block with "type = MODIFICATOR\_COORDINATE", index (will be described later)
and start and size (or end) 2D vectors.

\begin{verbatim}
{
  type = MODIFICATOR_COORDINATE

  index = 0

  start_x = 0
  start_y = 0

  size_x = 40
  size_y = 40
}
\end{verbatim}

This example describes an area which begins at (0,0) and size 40x40 pixels.

\section{Basic blocks}

An atomic part of the file is a block inside compound braces. It describes one 
atomic unit inside generator or some generator values. Each block must
contain its name and type.

\begin{verbatim}
{
  name = generator
  type = GENERATOR_MESH

  /* All generator params come here
  */
}
\end{verbatim}

Blocks can be nested, like this one:

\begin{verbatim}
/* Describes pixel generator and its color definition
*/
{
  name = pixel_point
  type = MODIFICATOR_POINT_SINGLE

  {
    type = MODIFICATOR_POINT_SINGLE_COLOR
    color_center = 3b5528
  }
}
\end{verbatim}

All block examples bellow uses this format:

\begin{verbatim}
{
  /* First part contains block name and type:
  */
  name = block_name
  type = block_type

  /*
    Second part is a list of all posible properties,
    descriptions and default values:

    [property_type]  property_name  - property description
   
    If the property_type is an enumerated type, all 
    posibilies come here:
    
    VALUE_1     - a description of VALUE_1
    VALUE_2     - a description of VALUE_1
    VALUE_3     - a description of VALUE_1
  */  
  property_name = default_value_of_the_property  
}
\end{verbatim}

\section{Generator architecture}

Whole generator is designed as a modificator chain. There is one master (root)
modificator and it passes results to slave modificators. A last modificator 
in the chain writes results (color pixel, heights) directly to a generator target
(it can be mesh itself, mesh texture or something else).

\paragraph{Basic terms:}
\subparagraph{Modificators}
are atomic generator parts specialized to one task. Each modificator 
is configured by properties (from the data file), coordinates (from the data 
file and/or previous modificator) and parameters (from previous modificator) and
pastes its results (coordinates, properties, parameters) to another modificator. 

For instance there is a modificator which generates a line and it 
pastes the results (coordinates for each single point which lies on the line) 
to another modificator which draws them.

\subparagraph{Targets}
are "final" modificators which transforms the results to geometry (mesh) or texture. 

\subparagraph{Generator}
launches one or more modificators and specify which targets are used.  
An output of generator is a complete 3D object with material and texture. 

Generator itself can be used as a modificator so
if we take the line modificator from previous example, 
the line modificator -> pixel modificator -> texture target chain will 
generate single pixels to texture, but line modificator -> generator 
chain will generate complete 3D objects on given coordinates.

\subparagraph{Generator launcher}
launches generators.

\section{Generators}

\subsection{Generator launcher}

Generator launcher defines which generators are performed and their order. 
It can be only one in the whole data file.

\begin{verbatim}
{
  /* Launcher name and type
  */
  name = generator_launcher_name
  type = GENERATOR_LAUNCHER

  /* Performed generators. 
  */
  generator_mesh = first_generator
  generator_mesh = second_generator
  generator_mesh = third_generator
}
\end{verbatim}

\subsection{Generator}

Generator defines which modifiators are launched, their targets and order. 
There can be as many generators as you want in a data file and 
are distinguished by their names:

\begin{verbatim}
/* A simple generator
*/
{
  /* Generator name and type
  */
  name = generator_name
  type = GENERATOR_MESH

  /*
    Modificator name and its target:
    
    [string]          modificator - a name of performed modificator
    [enumerated type] modificator_target - its target
    
    Modificator targets can be:
    
    TEXTURE         - texture target (color or height)
    GEOMETRY        - heights in mesh geometry
    GENERATOR_MESH  - target is another generator
    AUX             - an auxiliary surface (color or height)
  */
}
\end{verbatim}

There is an example of some generator there: 

\begin{verbatim}
/* A simple generator
*/
{
  /* Generator name and type
  */
  name = generator_name
  type = GENERATOR_MESH

  /* First modificator name and its target
  */
  modificator = first_modificator
  modificator_target = TEXTURE
  
  /* Second modificator name and its target
  */
  modificator = second_modificator
  modificator_target = GEOMETRY
}
\end{verbatim}

\subsection{Generated object parameters}

A 3D object generated by single generator is (for now) a flat mesh with
one big texture. If the texture is too big, it's sliced to smaller parts.
The object is described by mesh, material and texture block.

\subsubsection{Mesh params}

Describes generated mesh parameters like type, size and so on:

\begin{verbatim}
{
  name = mesh_name
  type = MESH_PARAMS

  /*
    [enumerated value]  mesh_type
  
    Mesh types can be:
   
    MESH_LAND   - a flat land
    MESH_BUNCH  - a bunch of plates
    MESH_GRASS  - not implemented yet
    MESH_BUSH   - not implemented yet
  */
  mesh_type = MESH_LAND
  
  /*
    Mesh dimensions. All values are vectors.
  
    [vector] start - lesh location 
    [vector] diff  - a size of one segment 
    [vector] size  - segments num
  */  
  start = (0,0,0)
  diff = (1,1,1)
  size = (1,1,1)
  
  /*
    Parameters related to bunch:
    
    [int, interval]   bunch_slice_num
    [int, interval]   bunch_slice_segments
      
    [float, interval] bunch_slice_x_offset
    [float, interval] bunch_slice_z_offset
      
    [angle, interval] bunch_slice_falling
    [angle, interval] bunch_segment_falling
        
    [int]             bunch_slice_rotation_incemental
    [angle, interval] bunch_slice_rotation_range
    [angle, interval] bunch_slice_rotation_step
  */    
  bunch_slice_num = 6
  bunch_slice_segments = 1
  
  bunch_slice_x_offset = 0
  bunch_slice_z_offset = 0
  
  bunch_slice_falling = 0
  bunch_segment_falling = 0
  
  bunch_slice_rotation_incemental = 0
  bunch_slice_rotation_range = 180
  bunch_slice_rotation_step = 0
}
\end{verbatim}

\subsubsection{Material params}

Describes material of a generated mesh:

\begin{verbatim}
{
  name = test_material
  type = MATERIAL_PARAMS
  
  /*
    [boolean] transparent
    Transparent material are for bunches
  */
  transparent = 0
  
  /*
    [boolean] double_side
    Double sided material are used by bunches
  */
  double_side = 0  
}
\end{verbatim}

\subsubsection{Texture params}

Describes texture for a generated mesh.

\begin{verbatim}
{
  name = test_texture
  type = TEXTURE_PARAMS

  /*
    [vector]  texture_size
    [int]     texture_height
    [color]   background_color
    [int]     texture_alpha
  */
  texture_size = (512,512)  
  texture_height = 512
  background_color = (0,0,0)
  texture_alpha = 0
}
\end{verbatim}

\section{Generator targets}
\subsection{GEOMETRY target}
\subsection{TEXTURE target}
\subsection{GENERATOR\_MESH target}
\subsection{AUX target}

\section{Generator modificators}

\subsection{Modificators and Coordinates}

Each modificator is applied to an area which is restricted by "top" coordinates. 
Top coortinates are defined by master modificator or size of target 
(for a first modificator).

Those "top" coordinates are further modified by local (in modificator) 
coordinate configuration (for instance by randomization, size extension
and so on).

\subsection{A generic modificator}

This is a basic setup which is included in any modificator.
All properties are available in all modificators, 
although they do not have to implement all of them 
and some properties can have a different meaning.

\begin{verbatim}
{
  /*
    Basic modificator properties:
  
    [boolean] area_inverted
    [int]     pixel_size
    
    [int]     pixel_step
    [int]     pixel_step_x
    [int]     pixel_step_y
  
    [boolean] pixel_step_random
    [int]     pixel_step_random_min
    [int]     pixel_step_random_max
    
    [float]   pixel_color_density
    
    [boolean] probability_fade
    [float]   probability_fade_start
    [float]   probability_fade_stop
  
    [boolean] color_fade
    [float]   color_fade_start
    [float]   color_fade_stop
      
    [boolean] erode_border
    [float]   erode_factor
    
    [float]   size_variator_theshold
    [float]   size_variator_factor
  */    
    
  /*
    Mask properties:
    
    [string]  mask
  */
    
  /*
    Slave modificators:
    
    You can define up to five modificators for each class.
    
    [string] modificator_slave - It's called for each coordinate generated
                                 by this master modificator.    
    [string] modificator_pre   - It's called before modificator start and
                                 with top coordinates only.
    [string] modificator_post  - It's called when modificator finishes and
                                 with top coordinates only.
  */  

  /*
    Local coordinates
    
    Each basic setup may contain local coordinate setup. It's defined by nested 
    MODIFICATOR_COORDINATE block and is described in next chaper.
  */
}
\end{verbatim}

\subsection{Coordinate specification}

Defines a block with local coordinate configuration.

pict.

Top coordinates are defined by master modificator or modificator target
(for first modificator). Local coordinates are defined by MODIFICATOR\_COORDINATE
block. It defines operation between top and local coordinates, whether the local
ones are generated (randomized) or not and so forth. If there are more than one
MODIFICATOR\_COORDINATE block, the configured modificator is called for each local
coordinate.

A part of coordinates setup is in basic modificator block and the rest 
is in MODIFICATOR\_COORDINATE blocks:

\begin{verbatim}
{
  /*
    Basic modificator block
  */

  /*
    Local coordinates setup
    
    Defines how are the local coordinates combined with the top one.
  
    [aritmetic operation] coordinates_operation 
    
      Defines operation between top and local coordinates.
    
    [boolean]             coordinates_random
    
      If it's set to 1, local coordinates are generated by random number
      generator in boundaries given by coordinates with index 0 and 
      index 1 (see bellow).
      
    [int]                 coordinates_random_num
    
      Number of generated local coordinates.
    
    [enumerated type]     modificator_start
    [enumerated type]     modificator_size
      
      It defines parts of top coordinates (start and size parts) for current 
      coordinates_operation. It can be top coordinates from previous modificator 
      (COORD_CURRENT) or result of last top and local coordinates composition:
      
      COORD_CURRENT           - current top coordinates
      COORD_LAST_START        - result of last coordinate composition (start part)
      COORD_LAST_SIZE         - result of last coordinate composition (size part)
      COORD_LAST_START_SIZE   - result of last coordinate composition (start+size parts)
      
      It's userful for generating objects which 
      have to be connected (e.g. objects strips).
  */  
  coordinates_operation = OPERATION_SET
  coordinates_random = 0
  coordinates_random_num = 0
  modificator_start = COORD_CURRENT
  modificator_size = COORD_CURRENT
  
  /*
    Local coordinates blocks
    
    There can be one or many of those blocks and each of them defines
    one local coordinate.
  */
  {
    type = MODIFICATOR_COORDINATE
    
    /*
      [vector] start  - coordinate start
      [vector] size   - coordinate size
      [int]    index  - coordinate index (used by randomized local coordinates)
    */
  }
}
\end{verbatim}

\subsection{Modificator parameters}

Parameters are float point values in <0,1> range which are passed between
modificators on parameter stack. If a modificator generates any parameter, the
parameter is added on top of the parameter stack. If a modificator does not
emit any parameter the stack is passed without modification.

The parameters and are typically used by simple point modificator 
for color/height generation and so on. For instance, there's a fractal
modificator which generates a height map. The fractal modificator calls
a slave modificator (simple point modificator for instance) for each generated 
pixel and as parameter passes pixel height. So the slave pixel modificator can 
draw pixels by color adjusted by pixel height.

\subsubsection{Modificator parameters type}

{\bf Modificator parameters} are defined by modificator parameter enum type:

\begin{verbatim}
  PARAM_PREV_0
  PARAM_PREV_1
  PARAM_PREV_2
  PARAM_PREV_3
  PARAM_PREV_4

    Those parameters refer to parameter stack. PARAM_PREV_0 is a parameter
    on top of the stack, PARAM_PREV_1 is a second one and so on.
  
  PARAM_SCATTER
  PARAM_SCATTER_HALF
  
    Parameters generated by random number generator. PARAM_SCATTER is from <-1,1>
    range and PARAM_SCATTER_HALF is from <0,1>.
    
  PARAM_HEIGHT_MAP
  PARAM_HEIGHT_MESH
  
    Parameters are extracted from attached mesh/heightmap. Not implemented yet.
    
  PARAM_HEIGHT_MAP_NORMAL
  PARAM_HEIGHT_MESH_NORMAL
  
    Parameters are normal vectors from attached mesh/heightmap. Not implemented yet.
  
  DEFAULT
  
    Default parameter, it's alias for PARAM_SCATTER_HALF.  
\end{verbatim}

\subsection{Point modificators}

Point modificators are designed as last modificators and usually write
data directly to targets (height to mesh geometry or color/heights to texture). 

\subsubsection{Single point modificator}

Single point modificator writes to target (slave modificator
or texture target) only one single point. Its size is always 1x1 so for instance
if it gets (20,20) -> (100,100) coordinate from master modificator, it writes
only single pixel to (20,20) with (1,1) size. Pixel\_size property is ignored 
by this modificator.

The single point modificator consists from basic setup in main block and
sub blocks. The sub blocks define particular color/height operations and 
are subsequently applied to a temporary color/height value. Number of 
color/height sub blocks is not limited.

This temporary color is get from modificator target, goes through sub blocks
and is applied back to modificator target (as color/height to texture or
height to mesh geometry). 

Single point modificator block contains:

\begin{verbatim}
{
  name = some_modificator_name
  type = MODIFICATOR_POINT_SINGLE

  /*    
     [enumerated type]  generator_type
     
       Generator used for pixel randomization:
       
       GENERATOR_GAUSS
       GENERATOR_RAND
      
     [boolean]          generator_separated
     
       For each cycle in color/height box (see bellow) is
       generated a new random value.
  */
  generator_type = GENERATOR_GAUSS
  generator_separated = 0
    
  /*      
     [aritmetic operation]  color_operation
     
       Color operation between target and color pixels 
       generated by this modificator.
     
     [bool]                 color_blend
     
       Blend the generated pixels?
  */      
  color_operation = SET
  color_blend = 0
  
  /*      
    [aritmetic operation]  height_operation;
    
       Height operation between target and heights
       generated by this modificator.
    
   */
  height_operation = SET

  /*
    Generated colors can be crop to this range.
  
    [boolean]               color_borders
    [color]                 color_border_min
    [color]                 color_border_max
   */
  color_borders = 0
  color_border_min = (0,0,0)
  color_border_max = (255,255,255)
  
  /*
     Color tables 
     
     Color table can define a colors which can be used for color generation. 
     For instance you can take a picture and generate pixels with colors from
     the image. If the color table is active, for each generated
     color is located the nearest color in the image (in RGB) and the nearest 
     color is used as a result instead of the generated one.
     
    [string]                color_table
      
      Image file (png, jpg,...) witch will be used for color table composition. 
      Final generated colors are altered with colors from this table.
    
    [string]                color_table_center
    
      Image file (png, jpg,...) witch will be used for color table composition. 
      Center colors (from each color box) are altered with colors from this table.
      
    [string]                color_table_delta    
    
      Image file (png, jpg,...) witch will be used for color table composition. 
      Delta colors (from each color box) are altered with colors from this table.
  */
  
  /*
    Color sub block describes single color operation.
  */
  {
    type = MODIFICATOR_POINT_SINGLE_COLOR
    [...]
  }
  
  /*
    Height sub block describes single height operation.
  */
  {
    type = MODIFICATOR_POINT_SINGLE_HEIGHT
    [...]
  }
}
\end{verbatim}

\subsubsection{Single point modificator - color sub block}

Color sub blocks defines single color operation and its result
is a single color which is applied to the temporary one.

\begin{verbatim}
{
  type = MODIFICATOR_POINT_SINGLE_COLOR

  /*  
    [aritmetic operation]   final_operation
    [boolean]               final_blend
    [modificator parameter] final_blend_parameter
  */
  final_operation = SET
  final_blend = 0
  final_blend_parameter = DEFAULT
  
  /*
    [aritmetic operation]   color_operation
    [boolean]               color_blend
    [modificator parameter] color_blend_parameter
  */
  color_operation = ADD
  color_blend = 0
  color_blend_parameter = DEFAULT
  
  /*
    [color]                 color_center
    [float]                 color_center_scale
    [float]                 color_center_delta
    [modificator parameter] color_center_parameter
  */
  color_center = (0,0,0,0)
  color_center_scale = 1
  color_center_delta = (0,0,0,0)
  color_center_parameter = PARAM_SCATTER_HALF;
  
  /*
    [color]                 color_delta
    [float]                 color_delta_scale
    [modificator parameter] color_delta_parameter
  */
  color_delta = (0,0,0,0)
  color_delta_scale = 1
  color_delta_parameter = PARAM_SCATTER_HALF
  
  /*
    [color]                 color_min
    [color]                 color_max

    [color]                 color_center_min
    [color]                 color_center_max
    
    Shortcuts for color definition:

    color_center = color_min
    color_delta = (color_max-color_min)
    
    color_center = color_center_min
    color_center_delta = (color_center_max-color_center_min)
  */  
}
\end{verbatim}

Output of each color block is a simple color which is blend with
the previous one.

MODIFICATOR\_POINT\_SINGLE\_HEIGHT

\begin{verbatim}
{
  type = MODIFICATOR_POINT_SINGLE_COLOR

  /*    
    [float]                 height_center
    [float]                 height_delta
  */
  height_center = 0
  height_delta = 0

  /*
    [float]                 height_min
    [float]                 height_max
    
    Shortcuts for height definition:

    height_center = height_min
    height_delta = (height_max - height_min)
  */
  
  /*
    [modificator parameter] height_parameter
  */  
  height_parameter = PARAM_SCATTER_HALF
  
  /*  
    [aritmetic operation]   height_operation
  */
  height_op.op = ADD
  
  /*  
    [aritmetic operation]   final_operation
  */
  final_operation = SET
}
\end{verbatim}

It's similiar to color block but works with height (float point number
from 0 to 1).

\subsubsection{Extended point modificator}
MODIFICATOR\_POINT\_EXTENDED

Extended point modificator draws a point (circle)
from single pixels. The pixel generation is controlled by 
properties of generic modificator.

\begin{verbatim}
{
  name = some_modificator_name
  type = MODIFICATOR_POINT_EXTENDED
}
\end{verbatim}

\subsection{Rectangle modificator}
MODIFICATOR\_RECT

Rectangle modificator generates points in whole area defined by coordinates.

\begin{verbatim}
{
  name = some_modificator_name
  type = MODIFICATOR_RECT
}
\end{verbatim}

\subsection{Height modificators}

Heightmap modifiators are used for height or parameter generation.

\subsubsection{Height map modificator}

\begin{verbatim}
{
  name = some_modificator_name
  type = MODIFICATOR_HEIGHT_MAP
  
  /*
    // Load heighmap from file
    [string]    height_bitmap
    
    // Load heighmap from generator (from target)
    [string]    height_source
  */
  
    [boolean]   heightmap_intensity
  */
  heightmap_intensity = 1
  
  /*    
    [float]     height_multiplier
    [float]     height_shift
  */
  height_multiplier = 1
  height_shift = 0
  
  /*
    [float]     height_range_min
    [float]     height_range_max
  */
  
  /*  
    [boolean]   scale_target
    [int]       scale_width
    [int]       scale_height
  */
  heightmap_scale = 0
  height_scale_width = 0
  height_scale_height = 0
}
\end{verbatim}

\subsubsection*{Generated parameters:}

\begin{tabular}{|l||l|}
  Param & Meaning \\
  0 & relative pixel height \\
  1 & pixel height \\
  2 & pixel intensity \\
\end{tabular}

\paragraph*{Relative pixel height}
means that pixel height is clamped to {\bf<height\_range\_min, height\_range\_max>}
ranges and adjusted by {\bf height\_multiplier} and {\bf height\_shift}. 
The formula is:

\begin{verbatim}
  height_translated = (height - height_range_min) / (height_range_max - height_range_min)
  height = height_translated*height_multiplier + height_shift
\end{verbatim}

\paragraph*{Pixel height} is an absolute pixel height. 
If {\bf height\_range\_min = 0.5} and {\bf height\_range\_max = 0.8}
all pixels are at this range <0.5,0.8>.

\paragraph*{Pixel intensity} means that a normal vector is 
calculated and its dot-product with light vector 
is passed here. The light vector is (0,1,0) by default.

\subsubsection{Mid-point modificator}

Generates mid-point fractal.

\begin{verbatim}
{
  name = some_modificator_name
  type = MODIFICATOR_FRACTAL

  /*
  [float]             fractal_hurst

  [float]             fractal_delta
  [float]             fractal_base
  
  [int]               limited_iteration
  [float]             limited_iteration_value
  
  [float]             correction_center
  [float]             correction_border
  
  [int]               filter_back
  [float]             border_start
  [float]             perturbation
  
  [int]               pixel_fill
  [int]               pixel_distance
  
  [int]               pixel_filter
  [int]               pixel_filter_num

  [enumerated type]   interpolation
  [enumerated type]   interpolation_first
  [enumerated type]   interpolation_second
  
  [int]               interpolation_border      
  [int]               generation_border
  */  
}
\end{verbatim}

\subsubsection{Perlin noise modificator}

Generates perlin noise.

\begin{verbatim}
{
  name = some_modificator_name
  type = MODIFICATOR_PERLIN

  /*
  [int]               perlin_octaves
  [int]               perlin_octaves_start
  [float]             perlin_persistence
  */
}
\end{verbatim}

\subsection{Line modificators}
\subsubsection{Single line modificator}

\begin{verbatim}
{
  name = some_modificator_name
  type = MODIFICATOR_LINE

  /*
    [int]       line_size
    
    Draws line from start point to start+size point.
  */
  line_size = 1
  
}
\end{verbatim}

\subsubsection{Leaf modificator}

\begin{verbatim}
{
  name = some_modificator_name
  type = MODIFICATOR_LINE_LEAF

  /*
    [float, interval] leaf_start
    [float, interval] leaf_stop
  
    [float, interval] leaf_width
    [angle]           leaf_thread_angle  
  */
}
\end{verbatim}

\subsubsection{Crack modificator}

\begin{verbatim}
{
  name = some_modificator_name
  type = MODIFICATOR_CRACK

  /*
    [enumerated type]   crack_type
  
      DEFAULT   - starts at coordinates start
      CENTER    - starts at coordinates center

    [int]               crack_branches
    [int]               crack_angle_random
    [float]             direction_angle_range
    [float]             direction_treshold
  */
}
\end{verbatim}

\subsubsection{Network modificator}

\begin{verbatim}
{
  name = some_modificator_name
  type = MODIFICATOR_NET

  /*
    [int] brick_corners
        
    [int] brick_width
    [int] brick_height
        
    [float] brick_width_scatter
    [float] brick_height_scatter
        
    [int] brick_width_max
    [int] brick_height_max
        
    [int] brick_width_zip
    [int] brick_height_zip
        
    [float] brick_width_join_pobability
    [float] brick_width_join_pobability_multiplier
        
    [float] brick_height_join_pobability
    [float] brick_height_join_pobability_multiplier
  */
}
\end{verbatim}

\subsection{Bunch modificator}

\begin{verbatim}
{
  name = some_modificator_name
  type = MODIFICATOR_BUNCH

  /*
    [float]           height
    
    [float]           height_correction_center
    [float]           height_correction_left
    [float]           height_correction_top      
    
    [float, interval] corner_curvature
    
    [int, interval]   points
    [float, interval] lenght
    
    [float]           angle
    [int]             border
  */
}
\end{verbatim}

\subsection{Mask modificator}

\begin{verbatim}
{
  name = some_modificator_name
  type = MODIFICATOR_MASK

  /*
    [string]          bitmap      
    [enumerated type] mask_type
  
      MASK_BOOL
      MASK_COLOR
      MASK_HEIGHT
  */
}
\end{verbatim}

\subsection{TODO}
MODIFICATOR\_BITMAP
MODIFICATOR\_LIGHT
MODIFICATOR\_FILTER
MODIFICATOR\_GENERATOR\_MESH

- tabulka co ktery modifikator zere, co je jeho vystup (parametry)
(v tech modifikatorech?)
- popsat jednotlive polozky pod tim seznamem jako
vyuctove typy

- pozuivat subsubsection*{} - zvyraznene oddeleni bez pocitani

\end{document}
