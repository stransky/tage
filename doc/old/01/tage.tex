\documentclass[11pt]{article}
\usepackage{a4wide}

\begin{document}

\section{Tage datafile description}

This file describes tage input files which describes whole scene.

\section{Comments}

The input file uses standard C++ comments:

\begin{verbatim}
// One line comment

/*
  Two or more line comment
*/
\end{verbatim}

\section{Values}

All values set by:

\begin{verbatim}
 value_name = value
\end{verbatim}

"value" can be strings, numbers (hexa, integer, float-point), colors, vectors
and enumerated types.

\subsection{Numbers}

Numbers are standard numerical values and can have a decimal part.

\begin{verbatim}
 size = 10
 height = 1.1
\end{verbatim}

\subsection{Strings}

Strings don't use commas and can't contain spaces. String values are typically 
used for modificator/generator names.

\begin{verbatim}
 name = my_modificator_name
\end{verbatim}

\subsection{Colors}

Colors can be defined by who ways - by separated RGB values
(0-255) or by one hexadecimal digit (HTML color). For instance, 
we want to set R:33, G:25, B:7 to color\_center value:

\begin{verbatim}
 // by RGB:
 color_center_r = 33
 color_center_g = 25
 color_center_b = 7
 
 // by one number (RRGGBB)
 color_center = 211907
\end{verbatim}

See the \_r,\_g and \_b suffixes. They are 0 by default.

\subsection{Vectors}

Vectors are composed from two or three numbers and they
can be integer or floating point numbers. For instance, we want to 
set light\_position vector:

\begin{verbatim}
 light_position_x = -1
 light_position_y =  1
 light_position_z = -1
\end{verbatim}

See the \_x,\_y and \_z suffixes. They are 0 by default.

\subsection{Angles}

An angles are normal numbers (an angle in degrees), from 0 to 360. 
They are used in polar coordinates and so on.

\begin{verbatim}
 some_angle = 20.6
\end{verbatim}

\subsection{Enumerated types}

Enumerated types are values which can have some predefined values. 
They are typically used for blocks type descriptions, some types,
targets and so on.

\begin{verbatim}
// coordinate type
type = MODIFICATOR_COORDINATE

// set modificator_target to texture
modificator_target = TEXTURE

// set modificator_target to geometry
modificator_target = GEOMETRY
\end{verbatim}

\subsection{Intervals}

Some values can be set as interval. If a value is set as interval,
it means it can get any value from the border values. The border values
are marked as "\_min" and "\_max" suffixes.

\begin{verbatim}
 // Interval set by only one value so it's always 10
 angle = 10

 // Interval set by two border values, 
 // can be any value from 10 to 20
 angle_min = 10
 angle_max = 20
\end{verbatim}

\subsection{Coordinates}

Coordinates are 2D area which describes 
where a modificator is applied.  The coordinate is a whole 
block with "type = MODIFICATOR\_COORDINATE", index (will be described later)
and start and size (or end) 2D vectors.

\begin{verbatim}
{
  type = MODIFICATOR_COORDINATE

  index = 0

  start_x = 0
  start_y = 0

  size_x = 40
  size_y = 40
}
\end{verbatim}

\section{Basic blocks}

An atomic part of the file is a block inside compound braces. It describes one 
atomic unit inside generator or some generator values. Each block must
contain its name and type.

\begin{verbatim}
{
  name = generator
  type = GENERATOR_MESH

  // All generator params come here
}
\end{verbatim}

Blocks can be nested, like this one:

\begin{verbatim}
// Describes pixel generator and its color definition
{
  name = pixel_point
  type = MODIFICATOR_POINT_SINGLE

  {
    type = MODIFICATOR_POINT_SINGLE_COLOR    
    color_center = 3b5528
  }
}
\end{verbatim}

\section{Generator architecture}

Whole generator is designed as a modificator tree. There is one master (root)
modificator and it passes results to slave modificators. A last modificator 
in the chain writes results (color pixel, heights) directly to a generator target
(it can be mesh itself, mesh texture or something else).

\section{Scene description}

A generated scene is (for now) a flat mesh with one big texture. If the texture
is too big, is sliced to smaller parts. The scene is described by Mesh, 
Material and Texture blocks.

\subsection{Mesh params}

Describes generated mesh parameters like type, size and so on:

\begin{verbatim}
{
  name = test_mesh
  type = MESH_PARAMS

  /*
    Mesh types can be:
   
    MESH_LAND   - a flat land
    MESH_BUNCH  - a bunch of plates
    MESH_GRASS  - not implemented yet
    MESH_BUSH   - not implemented yet
  */
  mesh_type = MESH_LAND
  
  /*
    Mesh dimensions. All values are vectors.
  
    [vector] start - lesh location 
    [vector] diff  - a size of one segment 
    [vector] size  - segments num
  */  
    
  /*
    Parameters related to bunch:
    
    [int, interval]   bunch_slice_num
    [int, interval]   bunch_slice_segments
      
    [float, interval] bunch_slice_x_offset
    [float, interval] bunch_slice_z_offset
      
    [angle, interval] bunch_slice_falling
    [angle, interval] bunch_segment_falling
        
    [int]             bunch_slice_rotation_incemental
    [angle, interval] bunch_slice_rotation_range
    [angle, interval] bunch_slice_rotation_step
  */
}
\end{verbatim}

\subsection{Material params}

Describes material of generated mesh:

\begin{verbatim}
{
  name = test_material
  type = MATERIAL_PARAMS
  
  // [int] transparent
  // Transparent material are for bunches
  transparent = 0
  
  // [int] double_side
  // Double sided material are used by bunches
  double_side = 0  
}
\end{verbatim}

\subsection{Texture params}

Describes texture for the generated mesh (material).

\begin{verbatim}
{
  name = test_texture
  type = TEXTURE_PARAMS

  /*
    [vector]  texture_size
    [int]     texture_height
    [color]   background_color
    [int]     texture_alpha
  */
}
\end{verbatim}

\section{Generator targets}

\section{Generator modificators}
  
\section{Generic modificator}
  
\section{Generic modificator}

  \{"MODIFICATOR\_COORDINATE",      MODIFICATOR\_COORDINATE\_TYPE\},
  \{"MODIFICATOR\_MASK",            MODIFICATOR\_MASK\_TYPE\},  
  \{"MODIFICATOR\_POINT\_SINGLE",    MODIFICATOR\_POINT\_SINGLE\_TYPE\},
  \{"MODIFICATOR\_POINT\_SINGLE\_COLOR",MODIFICATOR\_POINT\_SINGLE\_COLOR\_TYPE\},
  \{"MODIFICATOR\_POINT\_SINGLE\_HEIGHT",MODIFICATOR\_POINT\_SINGLE\_HEIGHT\_TYPE\},
  \{"MODIFICATOR\_POINT\_EXTENDED",  MODIFICATOR\_POINT\_EXTENDED\_TYPE\},
  \{"MODIFICATOR\_RECT",            MODIFICATOR\_RECTANGLE\_TYPE\},
  \{"MODIFICATOR\_LINE",            MODIFICATOR\_LINE\_TYPE\},
  \{"MODIFICATOR\_LINE\_LEAF",       MODIFICATOR\_LINE\_LEAF\_TYPE\},
  \{"MODIFICATOR\_BUNCH",           MODIFICATOR\_BUNCH\_TYPE\},
  \{"MODIFICATOR\_NET",             MODIFICATOR\_NET\_TYPE\},
  \{"MODIFICATOR\_HEIGHT\_MAP",      MODIFICATOR\_HEIGHT\_MAP\_TYPE\},
  \{"MODIFICATOR\_BITMAP",          MODIFICATOR\_BITMAP\_TYPE\},
  \{"MODIFICATOR\_FRACTAL",         MODIFICATOR\_FRACTAL\_TYPE\},
  \{"MODIFICATOR\_CRACK",           MODIFICATOR\_CRACK\_TYPE\},
  \{"MODIFICATOR\_LIGHT",           MODIFICATOR\_LIGHT\_TYPE\},
  \{"MODIFICATOR\_FILTER",          MODIFICATOR\_FILTER\_TYPE\},
  \{"GENERATOR\_MESH",              GENERATOR\_MESH\_TYPE\},
  \{"MODIFICATOR\_GENERATOR\_MESH",  MODIFICATOR\_GENERATOR\_MESH\_TYPE\},
  
\section{Generator launcher}

  GENERATOR\_LAUNCHER

\end{document}
